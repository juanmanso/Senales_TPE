
	\begin{equation}
		H_n(z)=\frac{1}{(1-p_n \cdot z^{-1})(1-p_n^* \cdot z^{-1})}
		\label{ec:transferencia}
	\end{equation}

	\begin{equation}
		p_n=e^{\frac{-2 \cdot \pi \cdot B}{F_s}} \cdot e^{j \cdot \frac{2 \cdot \pi \cdot F_n}{F_s}}
		\label{ec:polos}
	\end{equation}


\textit{Utilizando las ec. \ref{ec:transferencia} y \ref{ec:polos}, generar un modelo de tracto vocal para cada uno de los siguientes
conjuntos de valores de parámetros, que se corresponde con una vocal emitida por una locutora.}
\begin{table}[h!]
	\centering
	\begin{tabular}{*{9}{c}}
\toprule
	 	&$F_1$	&$B_1$	&$F_2$	&$B_2$	&$F_3$	&$B_3$	&$F_4$	&$B_4$	\\
\midrule
	a	&830	&110	&1400	&160	&2890	&210	&3930	&230	\\
	e	&500	&80	&2000	&156	&3130	&190	&4150	&220	\\
	i	&330	&70	&2765	&130	&3740	&178	&4366	&200	\\
	o	&546	&97	&934	&130	&2966	&185	&3930	&240	\\
	u	&382	&74	&740	&150	&2760	&210	&3380	&180	\\
\bottomrule
	\end{tabular}
\end{table}
\textit{Grafique diagrama de polos y ceros, y la respuesta en frecuencia del sistema para cada vocal, y
compare.}


