

\section{Ejercicio 4}	\label{ej4}
	\begin{flushleft}
	\textit{Reducir a la mitad la duración del fragmento musical utilizando el proceso de decimación.
No está permitido usar las funciones de matlab/octave que implementan la decimación
directamente. Recuerde que para evitar el aliasing es necesario realizar el adecuado filtrado
pasabajos. Utilice alguno de los métodos de diseño explicados en la teórica especificando el
criterio utilizado. Grafique el espectrograma de la nueva secuencia y compárela con la obtenida
en el Ejercicio 2.}
	\end{flushleft}

	\subsection{Decimación}
	Se recuerda que, cuando se decima por N una señal discreta que fue muestreada, al comprimir el espectro de frecuencias, la señal útil queda entre $-\pi/N$ y $\pi/N$ (en vez de $-\pi$,$\pi$). Todo lo demás tendrá aliasing al pasar por el conversor digital analógico (DAC). Por ende, como se pretende decimar por $N=2$, se aplica un pasabajos ideal entre $-\pi/2$ y $\pi/2$ (o en frecuencia $-F_S/4$, $F_S/4$). Dado que la función \texttt{fft()} da los valores entre 0 y $2\pi$, sólo se requiere eliminar desde $\pi/2$ hasta $\frac{3}{2}\pi$. Se utiliza el parámetro \texttt{'symmetric'} de la función \texttt{ifft()} para que sea posible reconstruir al poner ceros desde $\pi/2$ hasta $2\pi$ en el espectro de frecuencias.

		A continuación se expone la implementación utilizada:
		\begin{lstlisting}
aux=audio;

% Pasaje a frecuencia
NFFT=2^nextpow2(length(aux));
fft_mitad=fft(aux,NFFT);

% Frecuencias menores y mayores a |Fs/4| ==> =0
fft_mitad(fix(length(fft_mitad)/4)+1:end)=0;

% Reconstruccion
% Como se elimino desde 3/2*pi hasta 2*pi, queda asimetrica, por lo tanto
% se utiliza el parametro 'symmetric' para solucionarlo.
aux=ifft((fft_mitad),'symmetric');

% Decimacion.
mitad_audio=aux(1:2:end-1);
mitad_audio=mitad_audio(1:round(length(audio_input)/2));
		\end{lstlisting}


	\subsection{Comparación de espectrogramas}

	Como se sabe, al comprimir el dominio del tiempo, se expande el dominio de las frecuencias. Así, se ve en la Figura \ref{graf:ej4} compensando la frecuencia de muestreo, el eje de frecuencias se expande en 2 con respecto a la original (Figura \ref{graf:spec_ej2}).

		\graficarPNG{0.35}{4bis}{Espectrograma de la señal decimada con frecuencia de muestreo compensada.}{graf:ej4}
