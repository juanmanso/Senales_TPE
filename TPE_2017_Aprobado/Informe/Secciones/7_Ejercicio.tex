

\section{Ejercicio 7} \label{ej7}
	\begin{flushleft}
		\textit{Ahora se busca reducir la duración del audio a la mitad para lo cual se debe decimar el
espectrograma por columnas. Recuerde que para evitar el aliasing es necesario realizar el
adecuado filtrado pasabajos.}
	\end{flushleft}

	\subsection{Implementación}

	Para resolver este ejercicio, se realizó la misma rutina que la del Ejercicio \ref{ej4}	pero, al contar con los coeficientes de la transformada en el vector del \texttt{specgram()} no se requiere el pasaje a frecuencia ni el filtrado de dicho ejercicio (por la misma razón del ejercicio \ref{ej6}). Después de ello, se reconstruye con la función del Ejercicio \ref{ej5}.

	\subsection{Comparación de espectrogramas}


	\graficarPNG{0.35}{7}{Espectrograma de la señal resultante de la decimación de las columnas de S.}{graf:ej7}

	Al igual que en el Ejercicio \ref{ej6}, el espectograma es similar al del Ejercicio \ref{ej5} exceptuando la longitud del mismo. Utilizando la nueva implementación que suaviza las uniones, el audio contiene los mismos problemas que el Ejercicio anterior.\\

	Merece ser mencionado que no se realizó un filtrado pasabajos de la señal, porque la interpolación es de muestras con \emph{dft} predeterminada previamente. Otra propuesta interesante sería interpretar a las filas del espectrograma como señales individuales. Así se podría realizar la \emph{dft} de dichas señales y aplicar el filtro \emph{antialiasing}. Dado que el dichas señales no fueron muestreadas por \emph{Nyquist} y su posible dependencia con las ventanas y overlap utilizados, aplicar dicho método puede resultar erroneo. Sin embargo es interesante el enfoque de analizar las variaciones de amplitud de una ventana de frecuencias como una señal temporal ordinaria.
