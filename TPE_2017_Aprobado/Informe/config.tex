\input{.command.tex}
% En el siguiente archivo se configuran las variables del trabajo práctico
%% \providecommand es similar a \newcommnad, salvo que el primero ante un 
%% conflicto en la compilación, es ignorado.

% Al comienzo de un TP se debe modificar los argumentos de los comandos


\providecommand{\myTitle}{TRABAJO PRÁCTICO ESPECIAL}
\providecommand{\mySubtitle}{Modificación del \textit{tempo} de un fragmento musical.}

\providecommand{\mySubject}{Señales y Sistemas (85.05)}
\providecommand{\myKeywords}{UBA, Ingeniería, Informe, Señales, Sistemas}

\providecommand{\myAuthorSurname}{Manso}
\providecommand{\myTimePeriod}{Año 2017 - 1\textsuperscript{er} Cuatrimestre}

% No es necesario modificar este %%%%%%%%%%%%%%
\providecommand{\myHeaderLogo}{header_fiuba}
%%%%%%%%%%%%%%%%%%%%%%%%%%%%%%%%%%%%%%%%%%%%%%%%

% Si se utilizan listings, definir el lenguaje aquí
\providecommand{\myLanguage}{matlab}

% Crear los integrantes del TP con el comando \PutMember donde
%%		1) Apellido, Nombre
%%		2) Número de Padrón
%%		3) E-Mail
\providecommand{\MembersOnCover}[0]
{		\PutMember{Manso, Juan} {96133} {juanmanso@gmail.com}
}

\providecommand{\myGroupNumber}{4}

\Pagebreaktrue		% Setea si hay un salto de página en la carátula
\Indextrue
\Siunitxtrue			% Si quiero utilizar el paquete, \siunixtrue. Si no \siunixfalse
\Listingstrue
\Keywordsfalse
\Putgroupfalse		% Habilita/Deshabilita el \myGroup en los headers
